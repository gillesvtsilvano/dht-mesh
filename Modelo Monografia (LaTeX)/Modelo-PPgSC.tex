% Pre-ambulo
\documentclass[a4paper, 12pt]{abnt}

\usepackage[brazil]{babel}
\usepackage[latin1]{inputenc}
\usepackage[T1]{fontenc}
\usepackage{dsfont}
\usepackage{amssymb,amsmath}
\usepackage{multirow}
\usepackage[alf]{abntcite}
\usepackage[pdftex]{color, graphicx}
\usepackage{colortbl}
\usepackage{url}
\usepackage{abnt-alf}
\usepackage{abntcite}
\usepackage{algorithm}
\usepackage{algorithmic}
%\usepackage{alg}
%\usepackage{hyperref}


% Redefinicao de instrucoes
\floatname{algorithm}{Algoritmo}
\renewcommand{\algorithmicrequire}{\textbf{Entrada:}}
\renewcommand{\algorithmicensure}{\textbf{Sa�da:}}
\renewcommand{\algorithmicend}{\textbf{fim}}
\renewcommand{\algorithmicif}{\textbf{se}}
\renewcommand{\algorithmicthen}{\textbf{ent�o}}
\renewcommand{\algorithmicelse}{\textbf{sen�o}}
\renewcommand{\algorithmicfor}{\textbf{para}}
\renewcommand{\algorithmicforall}{\textbf{para todo}}
\renewcommand{\algorithmicdo}{\textbf{fa�a}}
\renewcommand{\algorithmicwhile}{\textbf{enquanto}}
\renewcommand{\algorithmicloop}{\textbf{loop}}
\renewcommand{\algorithmicrepeat}{\textbf{repetir}}
\renewcommand{\algorithmicuntil}{\textbf{at� que}}
\renewcommand{\algorithmiccomment}[1]{\% #1}


% Definicao da lista de simbolos
% \simb[entrada na lista de simbolos]{simbolo}:
% Escreve o simbolo no texto e uma entrada na lista de simbolos.
% Se o parametro opcional e omitido, usa-se o parametro obrigatorio.
\newcommand{\simb}[2][]
{%
	\ifthenelse{\equal{#1}{}}
	{\addcontentsline{los}{simbolo}{#2}}
	{\addcontentsline{los}{simbolo}{#1}}#2
}
% Para aceitar comandos com @ (at) no nome
\makeatletter 
% \listadesimbolos: comando que imprime a lista de simbolos
\newcommand{\listadesimbolos}
{
	\pretextualchapter{Lista de s�mbolos}
	{\setlength{\parindent}{0cm}
	\@starttoc{los}}
}
% Como a entrada sera impressa
\newcommand\l@simbolo[2]{\par #1}
\makeatother


% Definicao da lista de abreviaturas e siglas
% \abrv[entrada na lista de simbolos]{abreviatura}:
% Escreve a sigla/abreviatura no texto e uma entrada na lista de abreviaturas e siglas.
% Se o parametro opcional e omitido, usa-se o parametro obrigatorio.
\newcommand{\abrv}[2][]
{%
	\ifthenelse{\equal{#1}{}}
	{\addcontentsline{loab}{abreviatura}{#2}}
	{\addcontentsline{loab}{abreviatura}{#1}}#2
}
% Para aceitar comandos com @ (at) no nome
\makeatletter 
% \listadeabreviaturas: comando que imprime a lista de abreviaturas e siglas
\newcommand{\listadeabreviaturas}
{
	\pretextualchapter{Lista de abreviaturas e siglas}
	{\setlength{\parindent}{0cm}
	\@starttoc{loab}}
}
% Como a entrada sera impressa
\newcommand\l@abreviatura[2]{\par #1}
\makeatother


% \listofalgorithms: comando que imprime a lista de algoritmos
\renewcommand{\listalgorithmname}{Lista de algoritmos}


% Hifeniza��o de palavras feita de forma incorreta pelo LaTeX
\hyphenation{PYTHON ou-tros}


% Inicio do documento
\begin{document}

	\frenchspacing
	
	% Capa (arquivo Includes/Capa.tex)
	% Capa
% Prote��o externa do trabalho e sobre a qual se imprimem as informa��es indispens�veis 
% � sua identifica��o.

% Especifica��o da capa
\begin{titlepage}
	\begin{center}
		
		% Cabe�alho (n�o deve ser modificado)
		% Cont�m o bras�o da Universidade, o logotipo do Departamento, al�m dos dados
		% relacionados � vincula��o do aluno (Universidade, Centro, Departamento e Curso)
		\begin{minipage}{2.3cm}
			\begin{center}
				\includegraphics[width=2.25cm, height=2.68cm]{Imagens/Brasao-UFRN.jpg}
			\end{center}
		\end{minipage}
		\begin{minipage}{11.15cm}
			\begin{center}
				\begin{espacosimples}
					{\small \ \\
                       \textsc{Universidade Federal do Rio Grande do Norte}		   			\\
							  \textsc{Centro de Ci�ncias Exatas e da Terra}					\\
							  \textsc{Departmento de Inform�tica e Matem�tica Aplicada}	   	\\
							  \textsc{Programa de P�s-Gradua��o em Sistemas e Computa��o}  	\\
                       \textsc{Mestrado Acad�mico em Sistemas e Computa��o}}   				\\
				\end{espacosimples}
			\end{center}
		\end{minipage}
		\begin{minipage}{2.3cm}
			\begin{center}
				\includegraphics[width=2.52cm, height=1.96cm]{Imagens/Logotipo-DIMAp.png}
			\end{center}
		\end{minipage}
			
		\vspace{6cm}
						
		% T�tulo do trabalho
		{\setlength{\baselineskip}%
		{1.3\baselineskip}
		{\LARGE \textbf{T�tulo do trabalho}}\par}
			
		\vspace{3cm}
			
		% Nome do aluno (autor)
		{\large \textbf{Nome completo do autor}}
						
		\vspace{6cm}
		
		% Local da institui��o onde o trabalho deve ser apresentado e ano de entrega do mesmo
		Natal-RN\\M�s (por extenso) e ano
	\end{center}
\end{titlepage}

	% Folha de rosto (arquivo Includes/FolhaRosto.tex)
	% Folha de rosto
% Cont�m os elementos essenciais � identifica��o do trabalho.

% T�tulo, nome do aluno e respectivo orientador e filia��o
\titulo{\Large{T�tulo}}
\autor{Nome completo do autor}
\orientador[Orientador]{\par Nome completo do orientador e titula��o}
\instituicao
{
	PPgSC -- Programa de P�s-Gradua��o em Sistemas e Computa��o\par 
	DIMAp -- Departamento de Inform�tica e Matem�tica Aplicada\par
   CCET -- Centro de Ci�ncias Exatas e da Terra\par
   UFRN -- Universidade Federal do Rio Grande do Norte
}
	
% Natureza do trabalho (n�o deve ser modificada)
\comentario
{
	Disserta��o de Mestrado  apresentada ao Programa de P�s-Gradua��o em Sistemas e Computa��o do Departamento de Inform�tica e Matem�tica Aplicada da Universidade Federal do Rio Grande do Norte como requisito parcial para a obten��o do grau de Mestre em Sistemas e Computa��o.\bigskip\\
   \textit{Linha de pesquisa}:\\Nome da linha de pesquisa
}
		
% Local e data
\local{Natal-RN}
\data{M�s e ano}
	
\folhaderosto	
	
	% Folha de aprovacao (arquivo Includes/FolhaAprovacao.tex)
	% Folha de aprova��o
\begin{folhadeaprovacao}
	\setlength{\ABNTsignthickness}{0.4pt}
	\setlength{\ABNTsignwidth}{10cm}
	
	% Informa��es gerais acerca do trabalho 
	% (nome do autor, t�tulo, institui��o � qual � submetido e natureza)
	\noindent 
	Disserta��o de Mestrado sob o t�tulo \textit{T�tulo} apresentada por Nome completo do autor e aceita pelo Programa de P�s-Gradua��o em Sistemas e Computa��o do Departamento de Inform�tica e Matem�tica Aplicada da Universidade Federal do Rio Grande do Norte, sendo aprovada por todos os membros da banca examinadora abaixo especificada:
		
	% Membros da banca examinadora e respectivas filia��es
	\assinatura
	{
		Nome completo do orientador e titula��o   			                  \\
		{\small Presidente}											          \smallskip\\ 
		{\footnotesize
			DIMAp -- Departamento de Inform�tica e Matem�tica Aplicada		   \\
		  	UFRN -- Universidade Federal do Rio Grande do Norte
		}
   }
      
   \assinatura
	{
      Nome completo do examinador e titula��o   			                  \\
		{\small Examinador}											          \smallskip\\ 
		{\footnotesize
			Departamento		\\
		  	Universidade
		}
   }   
   
   \assinatura
	{
      Nome completo do examinador e titula��o   			                  \\
		{\small Examinador}											          \smallskip\\ 
		{\footnotesize
			Departamento		\\
		  	Universidade
		}
	}
		
	\vfill
	
	\begin{center}
		Natal-RN, data da defesa (dia, m�s e ano).
	\end{center}
\end{folhadeaprovacao}
	
	
	% Dedicatoria (arquivo Includes/Dedicatoria.tex)
	\include{Includes/Dedicatoria}
	
	% Agradecimentos (arquivo Includes/Agradecimentos.tex)
	\include{Includes/Agradecimentos}
   
   % Epigrafe (arquivo Includes/Epigrafe.tex)
	\include{Includes/Epigrafe}
	
	% Resumo em l�ngua vernacula (arquivo Includes/Resumo.tex)
	\include{Includes/Resumo}
	
	% Abstract, resumo em l�ngua estrangeira (arquivo Include/Abstract.tex)
	\include{Includes/Abstract}
	
	% Lista de figuras
	\listoffigures

	% Lista de tabelas
	\listoftables
	
	% Lista de abreviaturas e siglas
	\listadeabreviaturas
	
	% Lista de s�mbolos
	\listadesimbolos
	
	% Lista de algoritmos (se houver)
	% Devem ser inclu�dos os pacotes algorithm e algorithmic
	% \listofalgorithms
	
	% Sum�rio
	\sumario

	% Parte central do trabalho, englobando os cap�tulos que constituem o mesmo
	% Os referidos cap�tulos devem ser organizados dentro do diret�rio "Cap�tulos"

	% Capitulo 1: Introdu��o (arquivo Includes/Introducao.tex)
	\include{Capitulos/Introducao}
	
	% Capitulo 2: Segundo cap�tulo (arquivo Includes/Capitulo2.tex)
	\include{Capitulos/Capitulo2}
	
	% Capitulo 3: Terceiro cap�tulo (arquivo Includes/Capitulo3.tex)
	\include{Capitulos/Capitulo3}
	
	% Capitulo 4: Quarto cap�tulo (arquivo Includes/Capitulo4.tex)
	\include{Capitulos/Capitulo4}
	
	% Capitulo 5: Quinto cap�tulo (arquivo Includes/Capitulo5.tex)
	\include{Capitulos/Capitulo5}
		
	% Consideracoes finais
	\include{Capitulos/Consideracoes}
	
	% Bibliografia (arquivo Capitulos/Referencias.bib)
	\bibliography{Capitulos/Referencias}
	\bibliographystyle{abnt-alf}
	
	% Ap�ndice A (arquivo Includes/ApendiceA)
	\include{Capitulos/ApendiceA}
	
	% Anexo A (arquivo Includes/AnexoA)
	\include{Capitulos/AnexoA}
	
	% P�gina em branco
	\newpage

\end{document}